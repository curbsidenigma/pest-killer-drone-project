\documentclass[11pt]{exam}
\usepackage[utf8]{inputenc}
\usepackage[margin = 1in]{geometry}

% Language packages
\usepackage[spanish]{babel}
\usepackage{csquotes}

% Figure packages
\usepackage{graphicx}

% Reference packages
\usepackage[
    backend = biber,
    style = apa,
]{biblatex}
\addbibresource{main.bib}

\begin{document}
    \begin{titlepage}
        \centering
        {\includegraphics[width = 4in]{pictures/itesm-logo.png}\par}
        \vspace{0.4in}
        {\bfseries\LARGE Instituto Tecnol\'ogico y de Estudios Superiores de Monterrey \par}
        \vspace{0.4in}
        {\scshape\Large Laboratorio Integral de Control Automático \par}
        {\Large Dra. Debbie Crystal Hernández Zárate \par}
        {\Large Dra. Marybeth Flores Vázquez \par}
        \vspace{1.2in}
        {\Large ``Sistema de Control para la Eliminaci\'on de Pulg\'on Amarillo, \textit{Melanaphis sacchari}, en Cultivos de Maíz por medio de un Dron Parrot Bebop 2'' \par}
        \vspace{1.2in}
        {\itshape\Large Proyecto \par}
        \vfill
        {\Large Autores: \par}
        {\Large Gerardo Dom\'inguez Ram\'irez\par}
        {\Large Claudia Vanessa Dorantes Villegas\par}
        {\Large Uziel Hernández Espejo\par}
        {\Large Carlos Diego Fernández\par}
        {\Large Emmanuel Ramírez Reyes\par}
        \vfill
        {\Large Diciembre 2022 \par}
    \end{titlepage}

    \header{\includegraphics[width = 1in]{pictures/itesm-logo.png}}{}{Diciembre 2022}
    \footer{}{P\'agina\ \thepage\ de \numpages}{}
    \headrule
    \footrule

    \section*{Resumen}
        [Texto]
    
    \section{Introducci\'on} \label{sec1}
    La tecnolog\'ia est\'a desempeñando su papel en la globalizaci\'on y los drones presentan un uso cada vez mayor en una amplia gama de disciplinas (\cite{nouacer-2020}). La gran demanda de estos dispositivos se debe a su capacidad para responder a las necesidades de las personas. La mayor\'ia de estos brindan a los una amplia visión por medio de c\'amaras que se pueden activar y usar casi en cualquier lugar y en cualquier momento (\cite{yaacoub-2020}). Por ello sus aplicaciones se encuentran en un amplio rango de áreas, siendo las más relevantes la salud, el ej\'ercito y la agricultura (\cite{ayamga-2021}).

    Las tres revoluciones industriales anteriores transformaron profundamente la industria agr\'icola de la agricultura autóctona a la agricultura mecanizada y la agricultura de precisi\'on reciente (\cite{liu-2020}). Los drones están creando una nueva revolución agr\'icola. Se estima que el tamaño de los drones en el mercado agrícola alcanzar\'a los miles de millones de d\'olares en los pr\'oximos años. Como editor del informe de investigaci\'on de la Organizaci\'on de las Naciones Unidas para la Agricultura y la Alimentación y la Uni\'on Internacional de Telecomunicaciones sobre ``UAV y agricultura'', el experto en informaci\'on Gerard Sylvester dijo que mientras los agricultores trabajan para adaptarse al cambio clim\'atico y enfrentar otros desaf\'ios, se espera que los drones ayuden a todo el sector agr\'icola. las empresas mejoran la eficiencia (\cite{ren-2020}).

    Una de las afecciones más comunes en los campos de cultivo son las plagas y enfermedades que conllevan a bajos rendimientos de producción. Los agricultores se han basado tradicionalmente en métodos manuales para identificar plagas y enfermedades, lo que consume mucho tiempo y es costoso (\cite{xing-2022}). El internet y la omnipresencia de los dispositivos móviles con cámara en drones fungen como una oportunidad para adquisición de imágenes conveniente y económica, así como el uso de modelos de aprendizaje profundo para reconocer plagas y enfermedades en el campo.

    En el caso de M\'exico, los métodos agrícolas tradicionales de pequeñas parcelas trabajadas por familias y pequeñas comunidades contin\'uan dominando en muchas regiones, especialmente en aquellas con grandes poblaciones ind\'igenas como la Meseta Sur (\cite{alvarez-2018}). Muchos a\'un subsisten gracias a la agricultura de autoconsumo y ganan dinero vendiendo los excedentes de cosecha en los mercados locales, especialmente en el centro y sur de México (\cite{negrete-2018}). La automatizaci\'on de la agricultura y la detección y control de riesgos a pequeña y gran escala es de colosal importancia, ya que aplicando tecnolog\'ias mecatr\'onicas a la agricultura ayudar\'ia a detonar la productividad en la agricultura mexicana.

    Es por ello que la presente investigaci\'on tiene como objetivo implementar la simulaci\'on de un sistema de control para la eliminaci\'on de pulg\'on amarillo, \textit{Melanaphis sacchari}, en cultivos de ma\'iz en una regi\'on del territorio mexicano. Esto con la finalidad de acelerar el proceso de identificaci\'on y reducir su costo. Se plantea simular el recorrido del dron a trav\'es del campo de cultivo realizando las capturas necesarias sin perder la trayectoria. En la Secci\'on \ref{sec2} se realiza una investigaci\'on exhaustiva de los sistemas existentes que implementan sistemas de control y realizan procesos de identificaci\'on. En la Secci\'on \ref{sec3} se explica la metodología del sistema de control junto con análisis matemático y trayectorias de exploración en el campo de cultivo. En la Secci\'on \ref{sec4} se presentan los resultados de la simulaci\'on. Y finalmente en la Secci\'on \ref{sec5} se presentan las conclusiones.

    \section{Antecedentes}\label{sec2}

    \section{Metodolog\'ia}\label{sec3}

    \section{Resultados}\label{sec4}

    \section{Conclusiones}\label{sec5}

    \printbibliography

\end{document}